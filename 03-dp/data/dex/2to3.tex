                   Dataplot Analysis of a
                2**3 Full factorial Design:
                     Defective Springs
 
 
We recommend the following as a bare minimum
Dataplot analysis of a 2**3 full factorial design.
The analysis consists of the following 7 steps:
 
   1. pre-view the data file
   2. read in the data
   3. generate an effects plot for main factors
   4. generate an effects plot for main factors & 2-term interactions
   5. generate a matrix of effects plots
   6. do a Yates analysis to estimate effects
   7. generate a contour plot of the most important 2 factors
 
 
Experimental Background
 
You are a manufacturer of metal springs and you
wish to reduce the number of defective springs per
batch.  You suspect that 3 factors are affecting
the number of defectives: the temperature of the
oven for making the steels, the carbon
concentration of the steel, and the temperature of
the quenching temperature after the steel is
removed from the oven.  You run a 2**3 full
factorial design (8 runs) in which
 
   Y  = response variable = % acceptable per batch
   X1 = oven temperature (1450 and 1600 degress F)
   X2 = carbon concentration (.5 and .7%)
   X3 = quenching temperature (70 and 120 degrees F)
 
Your objective of course to determine which of
these factors (and interactions) effect the %
acceptable springs, and ultimately to determine
the settings of the factors which maximize the
percent acceptable springs.
 
Dataplot has several hundred auxiliary data files
(ASCII) that have been provided for user perusal
and which contain a variety of interesting data
sets.  The names of the available Dataplot
auxiliary data files are in the file
 
   DATASETS.TEX
 
You can thus scan Dataplot's available auxiliary
data file names by entering any of the following 3
equivalent commands:
 
   VIEW DATASETS.TEX
   PREVIEW DATASETS.TEX
   LIST DATASETS.TEX
 
Hypothetical data from the above 2**3 defective
springs experiment has been placed in the Dataplot
auxiliary file BOXSPRIN.DAT.  This example and
data were drawn from the manuscript [Box &
Bisgaard, 19xx]
 
 
1. Preview the data file
 
You can view the contents of the ASCII data file
BOXSPRIN.DAT by entering:
 
   VIEW BOXSPRIN.DAT
 
The following will appear on your screen:
 
   <output 1 here>
 
Note that this ASCII file has 25 lines of non-data
header information (most Dataplot auxiliary files
have 25 lines of header information) plus 8 lines
of data.  The 25 header lines describe the
contents of the file and how to read ths file into
Dataplot.  The 8 lines of data consist of 4 data
columns:
 
   column 1 = response = % acceptable springs per batch
   column 2 = factor 1 = coded oven temperature
   column 3 = factor 2 = coded carbon concentration
   column 4 = factor 3 = coded quench temperature
 
 
2. Read in the data
 
You can read the 4 data columns in the file BOXSPRIN.DAT
into Dataplot by entering:
 
   SKIP 25                           ignore the 25 header lines
   READ BOXSPRIN.DAT Y X1 X2 X3      read the data in
 
This read will be free-format.  The default
separator of adjacent numbers on a file line is
one (or more) blanks.  Dataplot will read all data
in until down to the end-of-file.  The 4 data
columns in the file will be read into 4 data
columns (= variables = vectors) internal to
Dataplot.  We have chosen to name these 4 internal
variables as Y, X1, X2, and X3.  You can name them
anything you like by changing the READ command.
For example,
 
   READ BOXSPRIN.DAT PERCACC OVEN CARB QUENCH
 
would name the 4 internal variables PERCACC, OVEN,
CARB, and QUENCH.  In Dataplot, characters beyond
the first 8 in a name are ignored, thus choose
whatever names you like but make them unique
within characters 1 to 8.
 
Upon entering the original READ command (READ
BOXSPRIN.DAT Y X1 X2 X3) the following feedback
message appears on your screen:
 
   <output 2 here>
 
 
3. Generate an effects plot for main factors
 
You can generate an effects plot (main factors
only) by entering
 
   DEX MEAN PLOT Y X1 X2 X3
 
The following will appear on your screen:
 
   <output 3 here>
 
The vertical axis is the mean response.  The horizontal axis
(1, 2, and 3) is the coded 3 factors:
 
   1 = factor 1 = coded oven temperature
   2 = factor 2 = coded carbon concentration
   3 = factor 3 = coded quench temperaTURE
 
The line on the plot above "1" connects the mean
response (xx) for factor 1 (oven temperature) at
the low setting (1450 degrees F), and the mean
response (xx) for factor 1 (oven temperature) at
the high setting (1600 degrees F).  The difference
between these 2 mean values is the least squares
estimate of the factor 1 effect.  In this case,
the estimated factor 1 effect is xx-yy = zz.
You thus see from the plot that
 
   mean response for oven temperature at 1450 degrees = xx
   mean response for oven temperature at 1600 degrees = xx
   estimated oven temperature effect = xx
 
   mean response for carbon concentration at .5% = xx
   mean response for carbon concentration at .7% = xx
   estimated carbon concentration = xx
 
   mean response for quench temperature at 70 degrees = xx
   mean response for quench temperature at 120 degrees = xx
   estimate quench temperature effect = xx
 
Main effects plots compares the relative
importance of the 3 factors and gives you a procedure
for constructing a ranked list of the 3 factors.  If a
line on a main effects plot is steep, the
estimated factor effect is large, and the factor
is important.  If a line on a main effects is
shallow, the estimated factor effect is small
(near zero), and the factor is unimportant.  In
this case, you see that
 
   1. factor 1 has the steepest line;
   2. factor 2 has the next steepest line;
   3. factor 3 has the least steep line;
 
thus
 
   1. factor 1 (oven temperature) is most important;
   2. factor 2 (carbon concentration) is next in importance;
   3. factor 3 (quench temperature) is least important.
 
The main effects plot also gives you a good method of
determining best settings for each of the 3 factors.
From the plot, for each factor, you choose the setting
which (of course) maximizes the % acceptable springs
Your best settings are therefore:
 
   1. factor 1 (oven temperature): 1600 degrees
   2. factor 2 (carbon concentration): .5%
   3. factor 3 (quench temperature): 120 degrees
 
Other considerations (to be discussed later) may
temper our choose for these best settings, but for
now these settings are clearly superior in terms
of improving our % acceptable springs.
 
4. Generate an effects plot for main factors & interactions
 
Main effects tell only part of the story in many physical phenomenona;
interactions are often also important.
Since this experiment involves 3 factors,
there are three 2-term interactions:
 
   1. oven temperature x carbon concentration
   2. oven temperature x quench temperature
   3. carbon concentration x quench temperature
 
and one 3-term interaction:
 
   oven temperature x carbon concentration x quench temperature
 
For 2-level experiments such as you ran above, you can extend the
above effects plot in a very simple fashion to tell if interactions
are important.
Since the 2 settings of each of the 3 factors
X1, X2 and X3 have been coded as -1 and +1, note that
you may form the 2-term interaction variables
X1*X2, X1*X3, and X2*X3
